\documentclass[a4paper, 12pt]{article}
\usepackage[danish]{babel}
\usepackage[utf8]{inputenc}
\usepackage[T1]{fontenc}
\usepackage{graphicx,amsmath,amssymb,hyperref}
\usepackage{tcolorbox}
\usepackage{amsthm}


\renewcommand{\vec}[1]{{\mathbf #1}}
\newcommand{\ora}{\overrightarrow} \newcommand{\mb}[1]{{\mathbf #1}}
\newcommand{\tc}[1]{\textcolor{#1}}
\theoremstyle{remark}
\newtheorem{Eksempel}{\textbf{Eksempel}}
\newtheorem{Interaktiv}{\textbf{Interaktivitet}}
\newtheorem{Opgave}{\textbf{Opgave}}
\begin{document}
\section*{Afstande udfra data}
Hvis data består af vægt og højde for nogen personer, er det ikke klart, hvad afstanden mellem to punkter $(v_1,h_1)$ og $(v_2,h_2)$ skal være. Altså, hvornår to punkter ligger tæt på hinanden. 
\begin{Eksempel}\label{ex:enheder og afstand}
Tre personer er givet som datapunkter (vægt, højde) $A=(70 kg, 165 cm)$, $B=(90 kg, 180 cm)$, $C=(80 kg, 190 cm)$.

Bruger vi Pythagoras på tallene, der står her, er afstanden mellem $A$ og $B$ $\sqrt{20^2+15^2}=25$. Afstanden mellem $A$ og $C$ er $\sqrt{10^2+25^2}=\sqrt{725}\simeq 27$ og mellem $B$ og $C$ $\sqrt{10^2+10^2}=\sqrt{200}\simeq 14$. Der er altså længst fra $A$ til $C$.

Skifter vi enhed og udtrykker højden i meter $A=(70 kg, 1,\!65 m)$, $B=(90 kg, 1,\!80 m)$, $C=(80 kg, 1,\!90 m)$ er afstanden mellem $A$ og $B$ cirka 20. Mellem $A$ og $C$ cirka $4$ og mellem $B$ og $C$ cirka $10$. Der er nu længst fra $A$ til $B$.

Det er ikke ret smart. Skal man finde de nærmeste naboer, er svaret tilsyneladende som vinden blæser. 

\end{Eksempel}

Selv hvis begge variable er i samme enhed, kan Pythagoras brugt med hovedet under armen være uheldigt:
\begin{Eksempel}
Vi har data for, hvor meget familierne $A$, $B$ og $C$  bruger på bolig og på mælk om måneden. Begge variable kan være i kroner. Hvis $A$ bruger $(7.500 kr, 200 kr)$ og $B$ bruger $(7.500 kr, 1.700 kr)$, mens $C$ bruger $(6.000 kr, 200 kr)$, er afstanden udregnet med Pythagoras den samme fra $A$ til $B$ som fra $A$ til $C$, men vi vil nok mene, $B$ afviger mere fra $A$ end $C$ gør, fordi mælkeforbruget i familie $B$ er usædvanligt.

Har vi data for mange familier, kan vi kvantificere ideen om, hvad der er usædvanligt og bruge det til at lave en mere passende afstand. 
\end{Eksempel}

\subsection*{Første naive tilgang - Min-Max-skalering:}

Data er punkter $(x_i,y_i)$ i planen. $x$-værdierne ligger mellem $a$ og $b$, mens $y$-værdierne ligger mellem $c$ og $d$. TEGNING af datapunkter.

Ideen er nu, at vi skalerer, så afstandene langs $x$-aksen får samme vægt som afstande langs $y$-aksen. 

Afstand fra $(x_1,y_1)$ til $(x_2,y_2)$ er $\sqrt{(\frac{x_2-x_1}{b-a})^2+(\frac{y_2-y_1}{d-c})^2}$

Overvej, at man får samme effekt, samme afstand, hvis man erstatter hver punkt $(x_i,y_i)$ med med $$(x_i,y_i)_{Norm}=\left(\frac{x_i-a}{b-a}, \frac{y_i-c}{d-c}\right)$$

\subsection*{Mindre naivt, mere bøvlet. Feature scaling}
Skalering langs akserne, hvor man bruger data til at bestemme skaleringen, kaldes også "feature scaling", når vi arbejder med perceptroner eller neurale netværk. Hvis det data, der skal læres fra - træningsdata - er $(x_1,y_1), (x_2,y_2),\ldots, (x_n,y_n)$, så skalerer vi langs førsteaksen ved
\begin{enumerate}
\item Udregn et estimat for middelværdien af $x$:
$$\bar{x}=\frac{\Sigma_{i=1}^nx_i}{n}$$
\item og et estimat for denne variabels spredning:
$$s_x=\sqrt{\frac{\Sigma_{i=1}^n(x_i-\bar{x})^2}{n-1}}$$
\end{enumerate}
Feature scaling af $x_i$ er da
$$\hat{x}_i=\frac{x_i-\bar{x}}{s_x}$$
Tilsvarende estimeres middelværdi og spredning for $y$ og feature scaling udregnes 
$$\hat{y}_i= \frac{y_i-\bar{y}}{s_y}$$

Euklidisk afstand mellem disse nye punkter $(\hat{x}_i,\hat{y}_i)$ og  $(\hat{x}_j,\hat{y}_j)$ er 

$$\sqrt{\left(\frac{x_i-\bar{x}}{s_x}-\frac{x_j-\bar{x}}{s_x}\right)^2 +\left(\frac{y_i-\bar{y}}{s_y}-\frac{y_j-\bar{y}}{s_y}\right)^2}=\sqrt{\left(\frac{x_i-x_j}{s_x}\right)^2+\left(\frac{y_i-y_j}{s_y}\right)^2}$$
Hvis vi sammenligner med den naive tilgang, er den ikke helt skæv. Der skal bare skaleres med $s_x$ i stedet for $b-a$ og med $s_y$ i stedet for $c-d$.
\begin{Opgave}
Se på INTERAKTIV ILLUSTRATION. Hvad er forskellen på den naive tilgang og \emph{feature scaling} i disse tilfælde? 
\end{Opgave}
\begin{Opgave}
Vis, at afstandene i såvel den naive, som den knap så naive tilgang giver metrikker. VINK: De er begge skaleringer, så man kan gøre det i et hug. 
\end{Opgave}
\begin{Opgave} Kan vi finde noget om middelværdi og spredning for eksemplerne med vægt, højde og det med bolig og mælk? Eller måske give 10 punkter? 
\end{Opgave}
\end{document}