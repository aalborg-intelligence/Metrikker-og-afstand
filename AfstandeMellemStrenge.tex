\documentclass[a4paper, 12pt]{article}
\usepackage[danish]{babel}
\usepackage[utf8]{inputenc}
\usepackage[T1]{fontenc}
\usepackage{graphicx,amsmath,amssymb,hyperref}
\usepackage{tcolorbox}
\usepackage{amsthm}


\renewcommand{\vec}[1]{{\mathbf #1}}
\newcommand{\ora}{\overrightarrow} \newcommand{\mb}[1]{{\mathbf #1}}
\newcommand{\tc}[1]{\textcolor{#1}}
\theoremstyle{remark}
\newtheorem{Eksempel}{\textbf{Eksempel}}
\newtheorem{Interaktiv}{\textbf{Interaktivitet}}
\newtheorem{Opgave}{\textbf{Opgave}}
\begin{document}
\subsubsection*{Afstande mellem ord.} 
 
Et ord er en følge eller en \emph{streng} af bogstaver eller tal \emph{hnaikgoh} (nej, det behøver ikke give mening). Det kunne være en DNA-sekvens, et ord i en tekst eller noget helt andet \footnote{Ofte gør man det desuden binært, så det er en streng af $0$ og $1$ såsom $00110110$ Det er fornuftigt nok, eftersom computere opererer med den slags strenge.}
 Længden af en streng er antallet af bogstaver.
\emph{Edit-afstande} er basalt set, hvor mange ændringer, man skal lave, for at komme fra den ene streng til den anden. Det afhænger naturligvis af, hvilke typer ændringer, man tillader. 
\begin{itemize}
\item Hammingafstanden mellem to lige lange strenge er Antallet af pladser, hvor de to strenge er forskellige. 
Afstand fra \emph{sne} til \emph{sno} er $1$. Afstand fra \emph{sne} til \emph{neg} er $3$. 
Det svarer til, at man må ændre et bogstav ad gangen. \emph{sne} $\rightarrow$ \emph{nne} $\rightarrow$ \emph{nee} $\rightarrow$ \emph{neg}.
\item Levenshteinafstanden har flere tilladte ændringer:  
Man må ændre bogstaver, som i Hamming, men man må også indsætte og fjerne bogstaver. 
Levenshteinfstanden er det mindste antal sådanne ændringer, man skal  lave for at nå fra det ene ord til det andet. 
Ordene/strengene behøver ikke have samme længde - man kan jo indsætte og fjerne bogstaver. 
\begin{itemize}
\item Afstand fra  \emph{sne} til \emph{sno} er $1$, ligesom Hammingafstanden. 
\item Afstand fra \emph{sne} til \emph{sned} er også $1$ - og her er Hammingafstanden slet ikke meningsfuld. Den er ikke defineret.
\item Afstand fra \emph{sne} til \emph{neg} er $2$ - via disse ændringer: \emph{sne} $\rightarrow$ \emph{ne}$\rightarrow$ \emph{neg}. Hammingafstanden er 3. 
\end{itemize}
Jo flere tilladte ændringer, jo kortere afstand. Der er algoritmer, der finder denne mindste vej mellem to ord - det er ikke helt så klart, hvordan man regner den ud, som for Hammingafstanden.
 Bemærk, at vi kunne have valgt \emph{sne} $\rightarrow$ \emph{sneg}$\rightarrow$ \emph{neg}, som også har to "moves". 
\item Damerau-Levenshteinafstanden er som Levenshtein, men man tillader nu også ombytning af to bogstaver, som står ved siden af hinanden. 
Hvis man skriver teskt på en telefon eller pc, er det let at bytte om på den måde. 
Hvis man så har en liste over ord, der giver mening, kan man opdage, at teskt ikke giver mening, men at ordet tekst ligger meget tæt på - afstand 1 i dette afstandsmål - afstand 2 i Hamming eller Levenshtein.
\end{itemize}
FIGUR: Graf med ord i hver knude og kanter svarende til edit moves. Måske interaktivt, så man kan se, at kanter forsvinder, hvis man går fra Levenshtein til Hamming. 
Afstand mellem DNA-strenge kan man se på med edit-afstande. Så det er bestemt ikke kun ord i sædvanlig forstand. 

\begin{Eksempel}
Afstande mellem navne: Navne som Peter, Pieter, Pietro, Petrus, Peder, Per, Pelle, Pekka, Peer, Petur, Pedro, Pierre, Pjotr, Pyotr, Petar eller måske Katarina, Katharina, Katrina, Katrine, Katrin, Cathryn, Kathryn, Catherine har samme oprindelse. Der er stor forskel på, hvor hyppigt, de optræder i forskellige lande. Overvej, om metrikkerne ovenfor kan bruges til at afsløre, hvor tæt på hinanden lande med Peter som hyppigst, er på lande med Pyotr. 


\end{Eksempel}
\end{document}