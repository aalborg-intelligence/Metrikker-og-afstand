\documentclass[a4paper, 12pt]{article}
\usepackage[danish]{babel}
\usepackage[utf8]{inputenc}
\usepackage[T1]{fontenc}
\usepackage{graphicx,amsmath,amssymb,hyperref}
\usepackage{tcolorbox}
\usepackage{amsthm}


\renewcommand{\vec}[1]{{\mathbf #1}}
\newcommand{\ora}{\overrightarrow} \newcommand{\mb}[1]{{\mathbf #1}}
\newcommand{\tc}[1]{\textcolor{#1}}
\theoremstyle{remark}
\newtheorem{Eksempel}{\textbf{Eksempel}}
\newtheorem{Interaktiv}{\textbf{Interaktivitet}}
\newtheorem{Opgave}{\textbf{Opgave}}
\begin{document}
\section*{Definition af en metrik- det abstrakte afstandsbegreb}

Man har ikke frit valg til at bestemme, hvad man vil bruge som afstandsmål. Hvis det skal give mening, skal man have en \emph{metrik} - afstanden skal opfylde nogle betingelser: 
\begin{tcolorbox}[title=Metrik]
En metrik på en mængde $M$ er en funktion $d$ fra $M\times M$  til $\mathbb{R}$ - altså en funktion, som tager to elementer  i $M$ og giver et reelt tal.

$d$ skal opfylde følgende for alle $p,q,r$ i $M$:
\begin{enumerate}
\item $d(p,q)\geq 0$. Med ord: Alle afstande er positive.
\item $d(p,p)=0$ og $d(p,q)=0$ hvis og kun hvis $p=q$.  Med ord: Afstanden fra et punkt til sig selv er $0$ og ingen andre afstande er $0$
\item $d(p,q)=d(q,p)$ Afstanden er \emph{symmetrisk}. Med ord: Der er lige så langt fra $p$ til $q$ som fra $q$ til $p$.
\item $d(p,q)+d(q,r)\geq d(p,r)$. \emph{Trekantsuligheden}. Med ord:  Der er mindst lige så langt fra $p$ til $ r$ via $q$, som direkte fra $p$ til $r$. 
\end{enumerate}
\end{tcolorbox}

Det er en meget kort definition. Og meget, meget generel. $M$ er en \emph{mængde} - der er en strengt logisk måde at gå til mængder på, men lad os her sige en samling af objekter, som vi også kalder elementer af mængden. Læg mærke til, at vi her bare graver problemet lidt længere ned i sandet - fejer det ind under gulvtæppet - for hvad er "objekter"?  Det kommer vi ikke nærmere her. 

Det er ret nemt at acceptere, at de tre krav er rimelige. Men er det nok? Og er det nu alligevel rimeligt? Hvad med symmetrien? Der er vel længere 10 km op ad bakke end 10 km ned ad bakke, hvis man tænker på arbejdsindsats. Så måske giver det ikke altid mening?\footnote{ Hvis funktionen $d$ opfylder 1,2,4, er det en \emph{quasimetrik}. Opfylder den 1,2,3, er det en \emph{semimetrik}. Opfylder den 1, 3 og 4, og første del af 2 ($d(p,p)=0$, men der kan være andre afstande, der er $0$)  er det en \emph{pseudometrik}.  Der findes såmænd også præmetrikker, metametrikker, pseudoquasimetrikker og sikkert andre - "falske metrikker". }

Definitionen af metrik som her, er den, vi bruger i matematik. Den har vist sig nyttig. Der er en skov af artikler og bøger, hvor man kan se, hvad man ved, når man har  en metrik. En mængde med en metrik kaldes et \emph{metrisk rum}.\footnote{Ordet "rum"  skal man ikke lægge for meget i. Det er ikke anden information i det end definitionen. Intution skal man være varsom med.}
\begin{Eksempel}
Den diskrete metrik: På en mængde $M$ er funktionen $d$ givet ved. 
\begin{itemize}
\item $d(p,p)=0$
\item Hvis $p\neq q$ er $d(p,q)=1$.
\end{itemize}
Det er en metrik - den opfylder definitionen ovenfor. Men det er ikke nogen specielt nyttig metrik. Alle elementer ligger lige tæt på alle andre, så der er ikke ny information - udover, om  to elementer er ens eller ikke.
\end{Eksempel}
\begin{Opgave}\label{Opg:Levensh} Vis, at Levenshteinafstanden giver en metrik. 
\begin{itemize}
\item Hvilken mængde er det mon en metrik på? Her kan man vælge - hvilke bogstaver må bruges? Vil I begrænse længden på de ord, der kan optræde? 
\item Overvej, at afstanden mellem to ord er længden af den (en - der kan være flere veje, som er lige lange)  korteste mulige vej fra det ene til det andet i et netværk (en graf) som på FIGUREN

\end{itemize}
Nu skulle det være til at indse, at de fire betingelser er opfyldt. 
\end{Opgave}
\begin{Eksempel}\label{ex:ikke-metrik} En elev er træt af kvadratrødder og tænker, at man vel kan lave sig en afstand i planen som følger:

 $D((x_1,y_1),(x_2,y_2))=(x_2-x_1)^2+(y_2-y_1)^2$. Problem: $D$ er ikke en metrik. Den opfylder ikke trekantsuligheden.

Hvordan kan man se det? Husk, at vi bare skal finde ét eksempel - tre punkter $p,q,r$, hvor trekantsuligheden ikke holder. Så har vi vist, at $D$ ikke er en metrik. 



Et konkret eksempel:  $p=(0,0)$, $q=(2,0)$, $r=(4,0)$. Afstand fra $p$ til $r$ er $4^2+0^2=16$. Afstanden fra $p$ til $q$ er $2^2=4$ og det samme gælder afstand fra $q$ til $r$, så $D(p,q)+D(q,r)=8$, mens $D(p,r)=16$.

Et andet eksempel, som ligner en rigtig trekant: $p=(0,0)$ $q=(2,1)$, $r=(4,0)$. Her er $D(p,q)=2^2+1^2=5$ og $D(q,r)=(4-2)^2+1^2=5$, så $D(p,q)+D(q,r)=10$, mens $D(p,r)=16$. Der er kortere at gå fra $p$ til $r$ via $q$ end at gå direkte. 


\end{Eksempel}
\begin{Opgave} Brug $D(p,q)$ fra Eksempel~\ref{ex:ikke-metrik}. Her regner vi på trekanter, hvor midterpunktet $q$ flyttes længere væk fra førsteaksen: $p=(0,0)$ $q=(2,y)$, $r=(4,0)$. Udregn $D(p,q)+D(q,r)$ og find det $y$, hvor $D(p,q)+D(q,r)=D(p,r)$. Hvad er vinklen $pqr$, når den ligning er opfyldt? Kunne man have indset det uden at regne?
\end{Opgave}
\end{document}