\documentclass[a4paper, 12pt]{article}
\usepackage[danish]{babel}
\usepackage[utf8]{inputenc}
\usepackage[T1]{fontenc}
\usepackage{graphicx,amsmath,amssymb,hyperref}
\usepackage{tcolorbox}
\usepackage{amsthm}


\renewcommand{\vec}[1]{{\mathbf #1}}
\newcommand{\ora}{\overrightarrow} \newcommand{\mb}[1]{{\mathbf #1}}
\newcommand{\tc}[1]{\textcolor{#1}}
\theoremstyle{remark}
\newtheorem{Eksempel}{\textbf{Eksempel}}
\newtheorem{Interaktiv}{\textbf{Interaktivitet}}
\newtheorem{Opgave}{\textbf{Opgave}}
\begin{document}
\section*{Afstande, nærmest, størst, mindst}
Når vi adskiller eller samler data bygger vi på en form for afstand. De $k$ \emph{ nærmeste} naboer er dem, der ligger tættest på i én eller anden forstand. 
Hvis det drejer sig om dem, hvis højder er tætte på hinanden eller måske dem, der vejer nogenlunde det samme, kan man se det for sig. 
Der er tal, man umiddelbart kan sammenligne. Men hvad med at sammenligne \emph{både} vægt og højde? Hvad betyder så mest? 
Er der lige langt mellem en person A, der vejer 80 kg og er 1,80 m høj og en anden, B,  der vejer 90 kg og er 2,00 m eller mellem A og C, der vejer 70 kg og er 1,60 m? 
Det er ikke klart, selvom vi da kan plotte de tre punkter i et (vægt, højde) koordinatsystem og endda bruge Pythagoras og få den samme afstand. 
Udregner man BMI, er $A$ tættere på $B$ end på $C$  \emph{personerne er muligvis ikke helt velvalgt...}
Det kommer nok også an på, hvad vi gerne vil udtale os om: Er de gode til at løbe langt? Eller hurtigt? Mere kompliceret bliver det, hvis vi også vil inddrage øjenfarve, skostørrelse eller måske, om de køber rigtig meget mælk.

Lad os begynde med afstand mellem punkter i planen uden at bekymre os om, hvad koordinaterne står for. Kald punkterne $p=(x_1,y_1)$ og $q=(x_2,y_2)$: 
\begin{itemize}
\item Euklidisk afstand mellem $p$ og $q$ er $\sqrt{(x_2-x_1)^2+(y_2-y_1)^2}$. Det er det, vi kender mest - det, der kommer ud af at bruge Pythagoras.
\item Manhattanafstanden er $|x_2-x_1|+|y_2-y_1|$. Det er afstanden, når man er tvunget til at bevæge sig langs akserne, som vi kender det fra vejene i mange amerikanske byer, herunder på Manhattan. Den kaldes også taxi-afstanden. 
\item Max-afstanden er maksimum af   $|x_2-x_1|$ og $|y_2-y_1|$. Den kaldes også skak-konge afstanden: Kongen i skak kan gå diagonalt eller langs de to akser. Et diagonalt move fra (a,b) til (a+k,b+k) tænkes at have længde k - som i skak. Skal man fra eksempelvis $(1,4)$ til $(3,7)$ kan skakkongen gå fra $(1,4)$ til $(3,6)$ - det stykke har længde $2$ og derefter fra $(3,6)$ til $(3,7)$ langs $y$-aksen - st stykke på længde $1$. Samlet afstand er $3$, maksimum af $|3-1|$ og $|7-4|$. 
\item Posthusafstanden\footnote{Den hedder også British Rail afstanden eller, hvis man er fransk, SNCF (Société Nationale des Chemins de fer Fran\c{c}ais) -afstanden. Man tænker sig, at man altid skal rejse via London (eller Paris) for at komme med tog fra et sted til et andet.} mellem $p$ og $q$ er $\sqrt{x_1^2+y_1^2}+\sqrt{x_2^2+y_2^2}$. Man kan tænke på, at der ligger et posthus i Origo og vi skal sende et brev fra $p$ til $q$. Det bliver transporteret fra $p$  til posthuset først og derefter fra posthuset til $q$.
\end{itemize}
Der er mange eksempler på afstande, som ikke umiddelbart er fysisk afstand:
\begin{Eksempel}
Afstande mellem ord. Lad os gøre det lidt lettere for os selv. 
Et ord er en følge eller en \emph{streng} af bogstaver \emph{hnaikgoh} (nej, det behøver ikke give mening).\footnote{Ofte gør man det desuden binært, så det er en streng af $0$ og $1$ såsom $00110110$ Det er fornuftigt nok, eftersom computere opererer med den slags strenge.}
 Længden af en streng er antallet af bogstaver.
Edit-afstande er basalt set, hvor mange ændringer, man skal lave, for at komme fra den ene streng til den anden. Det afhænger naturligvis af, hvilke typer ændringer, man tillader. 
\begin{itemize}
\item Hammingafstanden mellem to lige lange strenge er Antallet af pladser, hvor de to strenge er forskellige. 
Afstand fra \emph{sne} til \emph{sno} er $1$. Afstand fra \emph{sne} til \emph{neg} er $3$. 
Det svarer til, at man må ændre et bogstav ad gangen. \emph{sne} $\rightarrow$ \emph{nne} $\rightarrow$ \emph{nee} $\rightarrow$ \emph{neg}.
\item Levenshteinafstanden har flere tilladte ændringer:  
Man må ændre bogstaver, som i Hamming, men man må også indsætte og fjerne bogstaver. 
Levenshteinfstanden er det mindste antal sådanne ændringer, man skal  lave for at nå fra det ene ord til det andet. 
Ordene/strengene behøver ikke have samme længde - man kan jo indsætte og fjerne bogstaver. 
\begin{itemize}
\item Afstand fra  \emph{sne} til \emph{sno} er $1$, ligesom Hammingafstanden. 
\item Afstand fra \emph{sne} til \emph{sned} er også $1$ - og her er Hammingafstanden slet ikke meningsfuld. Den er ikke defineret.
\item Afstand fra \emph{sne} til \emph{neg} er $2$ - via disse ændringer: \emph{sne} $\rightarrow$ \emph{ne}$\rightarrow$ \emph{neg}. Hammingafstanden er 3. 
\end{itemize}
Jo flere tilladte ændringer, jo kortere afstand. Der er algoritmer, der finder denne mindste vej mellem to ord - det er ikke helt så klart, hvordan man regner den ud, som for Hammingafstanden.
 Bemærk, at vi kunne have valgt \emph{sne} $\rightarrow$ \emph{sneg}$\rightarrow$ \emph{neg}, som også har to "moves". 
\item Damerau-Levenshteinafstanden er som Levenshtein, men man tillader nu også ombytning af to bogstaver, som står ved siden af hinanden. 
Hvis man skriver teskt på en telefon eller pc, er det let at bytte om på den måde. 
Hvis man så har en liste over ord, der giver mening, kan man opdage, at teskt ikke giver mening, men at ordet tekst ligger meget tæt på - afstand 1 i dette afstandsmål - afstand 2 i Hamming eller Levenshtein.
\end{itemize}
Afstand mellem DNA-strenge kan man se på med edit-afstande. Så det er bestemt ikke kun ord i sædvanlig forstand. 
\end{Eksempel}
\begin{Eksempel}
Afstande mellem navne: Navne som Peter, Pieter, Pietro, Petrus, Peder, Per, Pelle, Pekka, Peer, Petur, Pedro, Pierre, Pjotr, Pyotr, Petar eller måske Katarina, Katharina, Katrina, Katrine, Katrin, Cathryn, Kathryn, Catherine har samme oprindelse. Der er stor forskel på, hvor hyppigt, de optræder i forskellige lande. Overvej, om metrikkerne ovenfor kan bruges til at afsløre, hvor tæt på hinanden lande med Peter som hyppigst, er på lande med Pyotr. 


\end{Eksempel}
De afstande, vi har set på her, er allesammen \emph{metrikker}. De opfylder nogen overordnede krav til afstandsmål. 
\begin{tcolorbox}[title=Metrik]
En metrik på en mængde $M$ er en funktion $d$ fra $M\times M$  til $\mathbb{R}$ - altså en funktion, som tager to elementer  i $M$ og giver et reelt tal.

$d$ skal opfylde følgende for alle $p,q,r$ i $M$:
\begin{enumerate}
\item $d(p,q)\geq 0$. Med ord: Alle afstande er positive.
\item $d(p,p)=0$ og $d(p,q)=0$ hvis og kun hvis $p=q$.  Med ord: Afstanden fra et punkt til sig selv er $0$ og ingen andre afstande er $0$
\item $d(p,q)=d(q,p)$ Afstanden er \emph{symmetrisk}. Med ord: Der er lige så langt fra $p$ til $q$ som fra $q$ til $p$.
\item $d(p,q)+d(q,r)\geq d(p,r)$. \emph{Trekantsuligheden}. Med ord:  Der er mindst lige så langt fra $p$ til $ r$ via $q$, som direkte fra $p$ til $r$. 
\end{enumerate}
\end{tcolorbox}

Det er en meget kort definition. Og meget, meget generel. $M$ er en \emph{mængde} - der er en strengt logisk måde at gå til mængder på, men lad os her sige en samling af objekter, som vi også kalder elementer af mængden. Læg mærke til, at vi her bare graver problemet lidt længere ned i sandet - fejer det ind under gulvtæppet - for hvad er "objekter"?  Det kommer vi ikke nærmere her. 

Det er ret nemt at acceptere, at de tre krav er rimelige. Men er det nok? Og er det nu alligevel rimeligt? Hvad med symmetrien? Der er vel længere 10 km op ad bakke end 10 km ned ad bakke, hvis man tænker på arbejdsindsats. Så måske giver det ikke altid mening?\footnote{ Hvis funktionen $d$ opfylder 1,2,4, er det en \emph{quasimetrik}. Opfylder den 1,2,3, er det en \emph{semimetrik}. Opfylder den 1, 3 og 4, og første del af 2 ($d(p,p)=0$, men der kan være andre afstande, der er $0$)  er det en \emph{pseudometrik}.  Der findes såmænd også præmetrikker, metametrikker, pseudoquasimetrikker og sikkert andre - "falske metrikker". }

Definitionen af metrik som her, er den, vi bruger i matematik. Den har vist sig nyttig. Der er en skov af artikler og bøger, hvor man kan se, hvad man ved, når man har  en metrik. En mængde med en metrik kaldes et \emph{metrisk rum}.\footnote{Ordet "rum"  skal man ikke lægge for meget i. Det er ikke anden information i det end definitionen. Intution skal man være varsom med.}
\begin{Eksempel}
Den diskrete metrik: På en mængde $M$ er funktionen $d$ givet ved. 
\begin{itemize}
\item $d(p,p)=0$
\item Hvis $p\neq q$ er $d(p,q)=1$.
\end{itemize}
Det er en metrik - den opfylder definitionen ovenfor. Men det er ikke nogen specielt nyttig metrik. Alle punkter ligger lige tæt på alle andre, så der er ikke ny information - udover, om  to punkter er ens eller ikke.
\end{Eksempel}
\begin{Opgave}\label{Opg:Levensh} Vis, at Levenshteinafstanden giver en metrik. 
\begin{itemize}
\item Hvilken mængde er det mon en metrik på? Her kan man vælge - hvilke bogstaver må bruges? Vil I begrænse længden på de ord, der kan optræde? 
\item Overvej, at afstanden mellem to ord er længden af den (en - der kan være flere veje, som er lige lange)  korteste mulige vej fra det ene til det andet i et netværk (en graf) som på FIGUREN

\end{itemize}
Nu skulle det være til at indse, at de fire betingelser er opfyldt. 
\end{Opgave}
\begin{Eksempel} I afsnittet om k-means clustering minimerer vi en sum af kvadratiske afstande, hvor afstand er Euklidisk - den med Pythagoras. Kvadratisk afstand er \emph{ikke} en metrik: 

Vi prøver i planen. $D((x_1,y_1),(x_2,y_2))=\|(x_2,y_2)-(x_1,y_1)\|^2=(x_2-x_1)^2+(y_2-y_1)^2$. Påstand: $D$ er ikke en metrik. Den opfylder ikke trekantsuligheden.

Hvordan kan man se det? Husk, at vi bare skal finde ét eksempel - tre punkter $p,q,r$, hvor trekantsuligheden ikke holder. Så har vi vist, at $D$ ikke er en metrik. 

Idé: Hvis $A,B,C$ er hjørnepunkter i en retvinklet trekant, hvor $C$ er den rette vinkel og sidelængderne er $a,b,c$ (modstående til vinkler med samme bogstav (OBS: TEGNING!)
så ved vi $D(B,C)+D(A,C)=D(A,B)$ da det netop er ligningen $a^2+b^2=c^2$. Nu fastholder vi sidelængderne $a$ og $b$, men gør vinklen ved $C$ større end $90^\circ$. Så bliver $c$ større og trekantsuligheden gælder ikke for vores afstandsmål $D$.

Et konkret eksempel: $C=(0,0)$ $A=(1,0)$, $B=(-1,1)$. $D(A,C)=1$, $D(B,C)=2$, $D(A,B)=5$, så der er "kortere" fra $A$ til $B$ via $C$ end direkte fra $A$ til $B$.
\end{Eksempel}
\section{ Afstand mellem DNA og RNA-strenge.} 
RNA er strenge med bogstaverne U (uracil), G (guanin), C (cytosin), A, (adenin). DNA har ikke U, men i stedet T (thymin) og DNA er dobbelt. U, G, C, A og T er nukleotider. 

Vi ser her på afstande mellem DNA (eller RNA) som bygger på antallet af mutationer for at nå fra den ene til den anden og desuden, hvor hyppige disse mutationer er - hvis man ved, en mutation sker ofte, er afstanden mellem en streng uden mutationen og en med mutationen ikke så lang, som hvis mutationen er meget sjælden. 
 Udover regler for, hvilke ændringer, man tillader, giver man derfor en omkostning ved ændringen - afstanden er ikke bare antal ændringer, men summen af, hvor "dyre" disse ændringer er.

\begin{Eksempel}
Afstand mellem DNA bruges til at analysere slægtskab og hvilke dyr, herunder mennesket, der nedstammer fra hvilke - fylogenetiske træer.\\ \url{ http://www.biotekunderviser.dk/sites/default/files/Handout Bioinformatik workshop Vejle 2.pdf}

I den sammenhæng kalder man skift mellem A og G eller mellem C og T for \emph{transitioner}\footnote{A og G er puriner, mens C og T er pyrimediner. Transition bytter en purin med en purin eller en pyrimedin med en pyrimedin.} . De fire andre mulige skift: Mellem A og C, mellem A og T, mellem G og T, mellem G og C, kaldes \emph{transversioner}. Transitioner er hyppigere mutationer end transversioner, så afstanden mellem $GATTACA$ og $GATTACG$ er mindre end mellem $GATTACA$ og $GATTACC$ . Den slags udskiftning af et bogstav (et basepar) kaldes en punktmutation.

Indel mutationer er indsætning ("In") eller fjernelse ("Del" for delete") af en eller flere basepar. Det er mindre hyppigt og svarer til længere afstand. I  KILDEN OVENFOR bruges følgende omkostninger og altså afstande mellem DNA-strenge; det er et valg - der er mange andre muligheder: 
\begin{itemize}
\item Transition: 1
\item Transversion: 2
\item Gap åbning: 9 (indsæt eller fjern præcis en base - altså et bogstav.
\item Gap forlængelse: 4 (indsæt eller fjern en base på samme sted, som er åbnet)
\end{itemize}
Man kan samle de to sidste og sige, at det koster $5+4L$ at indsætte eller fjerne en delstreng med $L$ bogstaver midt i et ord. (Overvej, at I forstår, det er samme regel.)

Vi tilføjer forlængelse/forkortelse: Det koster $4L$ at indsætte eller fjerne $L$ bogstaver i start eller slut af et ord.  Alt i alt:
\begin{itemize}
\item Transition: 1
\item Transversion: 2
\item Indsæt eller fjern delord med $L$ bogstaver midt i et ord: $5+4L$
\item Forlæng/forkort Indsæt eller fjern $L$ bogstaver i start eller slut af et ord: $4L$
\end{itemize}
\end{Eksempel}

Afstand mellem to strenge er så den kortest mulige måde, man kan komme fra den ene til den anden med de tilladte moves vægtet som her. 

Overvej, at samme argumenter som i Opgave \ref{Opg:Levensh} kan vise, at vi har lavet en metrik på mængden af DNA-strenge.

Et gratis move er at læse strengen bagfra, altså at vende den om. En DNA-streng har som udgangspunkt ikke en bestemt retning (ER DET MON RIGTIGT?)

\begin{Eksempel} 
Her bruger vi meget korte strenge. Det er naturligvis ikke realistisk. 

Vi vil finde afstanden fra $AGT$ til $ATG$. Der er mange muligheder for, hvordan man kan komme fra $AGT$ til $ATG$, altså hvordan mutationerne kunne have set ud. 
$$ AGT \to ATGT \to ATG$$
Indsæt $T$ mellem $A$ og $G$ og fjern så det sidste $T$. Det koster $9+4 =13$ - længden af den vej er $13$. En anden mulighed er 
$$AGT\to ATT \to ATG$$
Der er to punktmutationer og begge er transversioner (fra $T$ til $G$ eller omvendt), så det koster $2+2=4$. Det er faktisk den korteste vej, så afstanden er $4$. At den faktisk er kortest, kræver mere eftertanke. 

Havde vi brugt samme omkostning/vægt for alle tilladte ændringer, ville begge de to veje have samme længde. 

Hvad med fra $AGT$ til $TGA$? Jo, det er faktisk nemmere. Det er i virkeligheden samme DNA-sekvens; man har bare læst den fra den anden ende... 

\end{Eksempel}
Med lange strenge, som er ens på lange stykker, finder man afstande ved først at "aligne"\; - anbringe strengene, så de passer sammen på flest mulige pladser. Og derefter udregne afstande, men det er stadig ikke nemt - der skal algoritmer til:
\begin{Eksempel}
Streng 1: $TCGTAGG$ Streng 2: $TCTGTATCGA$
\\Første alignment:
$$\begin{matrix}T&C&G&-&-&-&T&A&G&G\\T&C&T&G&T&A&T&C&G&A\end{matrix}$$
Det koster indsættelse af $GTA$, $5+4x3=17$ to  transversioner $G\leftrightarrow T$ og $A\leftrightarrow C$ en transition $G\leftrightarrow A$. I alt $17+4+1=22$.\\
Hvis man vælger alignment 
$$\begin{matrix}T&C&-&-&-&G&T&A&G&G\\T&C&T&G&T&A&T&C&G&A\end{matrix}$$
er transversionen mellem $G$ og $T$ erstattet med en transition $G\leftrightarrow A$ og omkostningen falder med $1$ til $21$

Man indser ret let, at prisen for at klippe gør, at man ikke vil klippe to gange og  bruge 
$$\begin{matrix}T&C&-&G&-&-&T&A&G&G\\T&C&T&G&T&A&T&C&G&A\end{matrix}$$
hvor man kun sparer en enkelt transition, men hvad med 
$$\begin{matrix}T&C&-&G&T&A&-&-&G&G\\T&C&T&G&T&A&T&C&G&A\end{matrix}$$
Her er omkostningen $9$ for det første gap og $13$ for det andet. Og der er en transition i sidste plads $G\leftrightarrow A$, så omkostningen er $23$, men det er ikke helt så klart. I kan nok finde på eksempler, hvor det svarer sig at klippe flere steder. 
\end{Eksempel}



\section*{Afstande udfra data}
Hvis data består af vægt og højde for nogen personer, er det ikke klart, hvad afstanden mellem to punkter $(v_1,h_1)$ og $(v_2,h_2)$ skal være. Altså, hvornår to punkter ligger tæt på hinanden. 
\begin{Eksempel}
Tre personer har vægt, højde $A=(70 kg, 165 cm)$, $B=(90 kg, 180 cm)$, $C=(80 kg, 190 cm)$.

Bruger vi Pythagoras på tallene, der står her, er afstanden mellem $A$ og $B$ $\sqrt{20^2+15^2}=25$. Afstanden mellem $A$ og $C$ er $\sqrt{10^2+25^2}=\sqrt{725}\simeq 27$ og mellem $B$ og $C$ $\sqrt{10^2+10^2}=\sqrt{200}\simeq 14$. Der er altså længst fra $A$ til $C$.

Skifter vi enhed og udtrykker højden i meter $A=(70 kg, 1,\!65 m)$, $B=(90 kg, 1,\!80 m)$, $C=(80 kg, 1,\!90 m)$ er afstanden mellem $A$ og $B$ cirka 20. Mellem $A$ og $C$ cirka $4$ og mellem $B$ og $C$ cirka $10$. Der er nu længst fra $A$ til $B$.

Det er ikke ret smart. Skal man finde de nærmeste naboer, er svaret tilsyneladende som vinden blæser. OBS: PROBLEM i det videre (?) - vægt og højde er ikke uafhængige....

\end{Eksempel}

Selv hvis begge variable er i samme enhed, kan Pythagoras brugt med hovedet under armen være uheldigt:
\begin{Eksempel}
Vi har data for, hvor meget familierne $A$, $B$ og $C$  bruger på bolig og på mælk om måneden. Begge variable kan være i kroner. Hvis $A$ bruger $(7.500 kr, 200 kr)$ og $B$ bruger $(7.500 kr, 1.700 kr)$, mens $C$ bruger $(6.000 kr, 200 kr)$, er afstanden udregnet med Pythagoras den samme fra $A$ til $B$ som fra $A$ til $C$, men vi vil nok mene, $B$ afviger mere fra $A$ end $C$ gør, fordi mælkeforbruget i familie $B$ er usædvanligt.

Har vi data for mange familier, kan vi kvantificere ideen om, hvad der er usædvanligt og bruge det til at lave en mere passende afstand. 
\end{Eksempel}

\subsection*{Første naive tilgang:}

Data er punkter $(x_i,y_i)$ i planen. $x$-værdierne ligger mellem $a$ og $b$, mens $y$-værdierne ligger mellem $c$ og $d$. TEGNING af datapunkter - det skal nok ligne en ellipse, hvis $x$ og $y$ er uafhængige (?).

Ideen er nu, at vi skalerer, så afstandene langs $x$-aksen får samme vægt som afstande langs $y$-aksen. 

Afstand fra $(x_1,y_1)$ til $(x_2,y_2)$ er $\sqrt{(\frac{x_2-x_1}{b-a})^2+(\frac{y_2-y_1}{d-c})^2}$
\subsection*{Mindre naivt, mere bøvlet}
Noget om at centrere i middelværdi og skalere med variansen - eller spredningen, der er noget med en kvadratrod :-)


\section*{Clustering - klyngeanalyse}

Clustering er at samle datapunkter i grupper, klynger, så objekter i en klynge har mere til fælles med hinanden end med objekter i andre grupper. Har man et afstandsbegreb, der passer godt til det, vi mener med til fælles, kan man bruge det til clustering - objekter i samme klynge er tæt på hinanden, men længere fra punkter i de andre klynger. 
\subsection*{Hierarkisk clustering}
Her kender vi alle parvise afstande. Og ikke andet. 

Udfra den information laver vi et dendogram, hvor i første omgang par af datapunkter "mødes"\; i den højde, der svarer til deres afstand. Men der er mere: Hvornår skal datapunktet $p$ mødes med $qr$, som mødtes tidligere? Hvornår skal $pqr$ mødes med $ab$?  Det er \emph{linkage}-reglerne.
\begin{itemize}
\item Single linkage: $pqr$ mødes med $ab$ i den højde, hvor minimumsafstanden mellem de to grupper af punkter nås:\\ Minimum af $d(a,p),d(a,q), d(a,r), d(b,p), d(b,q), d(p,r)$
\item Complete linkage: $pqr$ mødes med $ab$, når den maksimale afstand mellem punkter i de to grupper er nået. \\Maksimum af $d(a,p),d(a,q), d(a,r), d(b,p), d(b,q), d(p,r)$
\item Middelafstand- average linkage: Når den gennemsnitlige afstand er nået. $\frac{1}{2\cdot 3}(d(a,p)+d(a,q)+ d(a,r)+ d(b,p)+ d(b,q)+ d(p,r))$
\end{itemize}
(OBS: Her skal være tegninger og diagrammer -dendrogrammer. Og eksempler på, hvad forskellen er på de forskellige linkagekrav)

\begin{Eksempel} Klyngeanalyse af DNA eller for eksempel mRNA giver anledning til dendrogrammer, som kaldes de phylogenetiske træer for de arter/sygdomme,... der svarer til den analyserede DNA. 

https://www.ncbi.nlm.nih.gov/pmc/articles/PMC2859286/

https://www.ncbi.nlm.nih.gov/pmc/articles/PMC6130602/
\end{Eksempel}
\subsection*{k-means clustering}
 Vores data er punkter med $d$ koordinater. Afstanden er Euklidisk. Vi vælger $k$, det antal clusters, det skal ende med. Målet er at opdele data i $k$ dele, $S_1, S_2,\ldots , S_k$ så den samlede gennemsnitlige kvadratiske afstand $$\Sigma_{i=1}^{k}\Sigma_{p,q\in S_i}\frac{1}{2|S_i|}\|p-q\|^2$$ indenfor de $k$ clusters er mindst mulig. 







\section*{Clustering og anbefalingsalgoritmer}

\emph{Clustering} er samling af datapunkter i "klumper", hvor punkterne i en klump ligner hinanden, men ikke ligner dem, der er i andre klumper. Det kan man bruge en metrik til, men der er andre måder, data kan ligne hinanden på. Når musiktjenester anbefaler musik, online boghandlere anbefaler bøger, online supermarkeder anbefaler grønsager etc. "Andre, der hører xxx hører også yyy." "Andre, der har set filmen xxx har også set filmen yyy", så baserer de anbefalingerne på en forståelse af, hvordan vi ligner andre kunder. 

Ligner handler her om, hvor mange film, vi har fælles med hinanden. Der er et stykke vej, til definitionen, men hold hovedet koldt, så går det:

Hyppige delmængder - frequent item set:

Hvis vi har forbrugere, der har købt/lyttet til/set som følger:
\begin{center}
\begin{tabular}{cc}
1 & \{ c,e,a\} \\
2 & \{ d,b,e\} \\
3 & \{ b,c,e,d\} \\
4& \{ c,e,d,a\} \\
5& \{b,e\} \\
6& \{c,d,a\} \\
7&\{b,e,a\} \\
8& \{b,c\} \\
9& \{c,e,d\} 

\end{tabular}
\end{center}
uddrager vi hyppige delmængder

\begin{center}
\begin{tabular}{cc}


\{a\} & hørt af 1,4,6,7 \\
\{b\} & hørt af 2,3,5,7,8 \\
\{c\}& hørt af 1,3,4,6,8\\
\{d\} & hørt af 2,3,4,6,8,9 \\
\{e\} & hørt af 1,2,3,4,5,7,9 \\
\{a,b\} & hørt af 7 \\
\{a,c\} & hørt af 1,4,6 \\
\{a,d\}& hørt af 4,6\\
\{ a,e\}& hørt af 1,4,7\\
\{b,c\} & hørt af 3, 8 \\
\{b,d\}& hørt af 2,3,8 \\
\{b,e\}& hørt af  2,3,5,7\\
\{c,d\}& hørt af 3,4,6,8\\
\{c,e\}& hørt af 1,3,4\\
\{d,e\}& hørt af 2,3,4,9\\
\{a,c,d\}& hørt af 4\\
\{a,c,e\}& hørt af 1,4 \\
\{b,c,d\}& hørt af 3\\
\{b,c,e\}& hørt af 3\\
\{b,d,e\}& hørt af 2,3 \\
\{c,d,e\} & hørt af 3,4,9 \\

\end{tabular}
\end{center}
Det løber ret hurtigt ud over det, man kan overskue. Men vi kan strukturere: Når \{c,d,e\} er set/hørt af  3,4,9 er det klart, at \{c,d\}, \{c,e\} og \{d,e\} også er hørt af  disse og muligvis andre. Desuden synes vi nok ikke, delmængder, der kun er hørt af én, er "hyppige delmængder" og vi er ikke interesserede i delmængder med kun ét element - vi vil jo anbefale noget nyt. Vi reagerer med to tiltag:
\begin{itemize}
\item  Vi sætter en minimal længde - lad os sige 2, så kun delmængder med mindst to elementer er med.  
\item En minimal support - hvor stor en andel af lytterne skal have denne delmængde til fælles. Her vælger vi 33\% - så mindst 3 skal have delmængden fælles.
\end{itemize}
Tilbage har vi 
\begin{center}
\begin{tabular}{cc}
\{a,c\} & hørt af 1,4,6 \\
%%\{a,d\}& hørt af 4,6\\
\{ a,e\}& hørt af 1,4,7\\
%\{b,c\} & hørt af 3, 8 \\
\{b,d\}& hørt af 2,3,8 \\
\{b,e\}& hørt af  2,3,5,7\\
\{c,d\}& hørt af 3,4,6,8\\
\{c,e\}& hørt af 1,3,4\\
\{d,e\}& hørt af 2,3,4,9\\

%\{a,c,e\}& hørt af 1,4 \\

%\{b,d,e\}& hørt af 2,3 \\
\{c,d,e\} & hørt af 3,4,9 \\
\end{tabular}
\end{center}



Nu vender vi det lidt om og vil gruppere lytterne/seerne. Vi vil lave clustering baseret på tabellen ovenfor. 

\end{document}

%%% Local Variables: 
%%% mode: latex
%%% TeX-master: t
%%% End: 
