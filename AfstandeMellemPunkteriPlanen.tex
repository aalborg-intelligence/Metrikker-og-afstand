\documentclass[a4paper, 12pt]{article}
\usepackage[danish]{babel}
\usepackage[utf8]{inputenc}
\usepackage[T1]{fontenc}
\usepackage{graphicx,amsmath,amssymb,hyperref}
\usepackage{tcolorbox}
\usepackage{amsthm}


\renewcommand{\vec}[1]{{\mathbf #1}}
\newcommand{\ora}{\overrightarrow} \newcommand{\mb}[1]{{\mathbf #1}}
\newcommand{\tc}[1]{\textcolor{#1}}
\theoremstyle{remark}
\newtheorem{Eksempel}{\textbf{Eksempel}}
\newtheorem{Interaktiv}{\textbf{Interaktivitet}}
\newtheorem{Opgave}{\textbf{Opgave}}
\begin{document}
\section*{Afstande mellem punkter i planen.}
Forskellige afstande mellem punkter i planen. Kald punkterne $p=(x_1,y_1)$ og $q=(x_2,y_2)$: 
\begin{itemize}
\item Euklidisk afstand mellem $p$ og $q$ er $\sqrt{(x_2-x_1)^2+(y_2-y_1)^2}$. Det er det, vi kender mest - det, der kommer ud af at bruge Pythagoras.
\item Manhattanafstanden er $|x_2-x_1|+|y_2-y_1|$. Det er afstanden, når man er tvunget til at bevæge sig langs akserne, som vi kender det fra vejene i mange amerikanske byer, herunder på Manhattan. Den kaldes også taxi-afstanden. 
\item Max-afstanden er maksimum af   $|x_2-x_1|$ og $|y_2-y_1|$. Den kaldes også skak-konge afstanden: Kongen i skak kan gå diagonalt eller langs de to akser. Et diagonalt move fra (a,b) til (a+k,b+k) tænkes at have længde k - som i skak. Skal man fra eksempelvis $(1,4)$ til $(3,7)$ kan skakkongen gå fra $(1,4)$ til $(3,6)$ - det stykke har længde $2$ og derefter fra $(3,6)$ til $(3,7)$ langs $y$-aksen - st stykke på længde $1$. Samlet afstand er $3$, maksimum af $|3-1|$ og $|7-4|$. 
\item Posthusafstanden\footnote{Den hedder også British Rail afstanden eller, hvis man er fransk, SNCF (Société Nationale des Chemins de fer Fran\c{c}ais) -afstanden. Man tænker sig, at man altid skal rejse via London (eller Paris) for at komme med tog fra et sted til et andet.} mellem $p$ og $q$ er $\sqrt{x_1^2+y_1^2}+\sqrt{x_2^2+y_2^2}$. Man kan tænke på, at der ligger et posthus i Origo og vi skal sende et brev fra $p$ til $q$. Det bliver transporteret fra $p$  til posthuset først og derefter fra posthuset til $q$.
\end{itemize}
\end{document}